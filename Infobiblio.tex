% !TeX encoding = UTF-8 Unicode
% !TeX root = GANXXX.tex
% !TeX spellcheck = es_ES
%%=========================================	
\addcontentsline{toc}{chapter}{Información sobre la bibliografía}
\chapter*{Información sobre la bibliografía}
Todos los documentos técnicos (Trabajo Fin de Grado, Tesis Doctoral, Artículo Técnico o Científico, Memoria, etc.) deben incluir una sección de bibliografía en la cual se hace un listado de todas las fuentes consultadas para realizar el trabajo. Es preciso otorgar el crédito a los trabajos realizados por otros y que se utilizan de algún modo u otro en el trabajo propio. Por esta razón, en el texto se deben incluir las referencias a las fuentes empleadas intentando incurrir en plagio, aún cuando ésta no sea nuestra intención. Se citará usando \lstinline!\cite{einstein}! \cite{einstein}.


Todas las fuentes utilizadas deberían ser referenciadas en nuestro trabajo. Estas fuentes pueden ser:

\begin{itemize}
	\item Un libro o capítulo de libro
	\item Un artículo de revista
	\item Un artículo de congreso
	\item Un manual técnico
	\item Un trabajo anterior (Fin de Grado, Fin de Máster, Tesis Doctoral, etc.)
	\item Un enlace virtual (Wikipedia, etc.)
\end{itemize}

Se recomienda usar para la bibliografía programas como ``JabRef'' o ``BibTex'' para gestionar las referencias. El estilo más extendido en publicaciones es el IEEEtran, que en \LaTeX $ $ se denomina \texttt{ieeetr}. Aparecen listados los trabajos a medida que se van citando, tal y como se puede comprobar \cite{knuthwebsite} en la siguiente página. Con este tipo de bibliografía, no aparecerán los trabajos hasta que éstos sean citados.

El siguiente código genera la bibliografía
\begin{lstlisting}
	\bibliographystyle{ieeetr}
	\addcontentsline{toc}{chapter}{\bibname}
	\bibliography{refsreal}  
\end{lstlisting}
